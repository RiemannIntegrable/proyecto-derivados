%% ==== CONFIGURACIÓN DE IDIOMA Y CODIFICACIÓN ====
\usepackage[T1]{fontenc}
\usepackage[spanish]{babel}
\spanishdecimal{.}

%% ==== PAQUETES MATEMÁTICOS ====
\usepackage{
  amsmath, 
  amssymb, 
  amsfonts, 
  yhmath, 
  amsthm, 
  mathtools, 
  mathrsfs, 
  cancel, 
  bigints, 
  fixmath,
  actuarialsymbol
}
\usepackage{makecell}

%% ==== PAQUETES PARA BIBLIOGRAFÍA ====
\usepackage[style=ieee]{biblatex}
\addbibresource{referencias.bib}

%% ==== PAQUETES PARA GRÁFICOS Y DIAGRAMAS ====
\usepackage{
  graphicx,
  tikz,
  tikz-cd,
  pgfplots,
  venndiagram,
  xcolor,
  forest,
  booktabs
}
\usepackage[all]{xy}

%% ==== PAQUETES PARA FORMATO Y ESTRUCTURA ====
\usepackage{
  lipsum, 
  multicol, 
  float, 
  multirow, 
  array, 
  tcolorbox, 
  anyfontsize, 
  xltxtra
}

%% ==== PAQUETES PARA CÓDIGO Y BIBLIOGRAFÍA ====
\usepackage{listings}
\usepackage{hyperref}
\usepackage{dirtree}

%% ==== PAQUETES PARA FUENTES ====
\usepackage{fontspec}

%% ==== PAQUETES PARA ENCABEZADO Y PIE DE PÁGINA ====
\usepackage{fancyhdr, lastpage}

%% ==== CONFIGURACIONES BÁSICAS DE PAQUETES ====
\pgfplotsset{compat=1.18}
\usepgfplotslibrary{external}
\tikzexternalize