\section{Marco teórico}

Una serie de tiempo es una serie de observaciones indexadas en el tiempo. Mediante modelos de serie clásicos buscan describir el comportamiento del proceso estocástico que subyace a los datos. Bajo condiciones de estacionaridad modelos como el ARMA logra capturar el comportamiento de la serie de tiempo. No obstante, en muchos modelos no se puede asumir estacionaridad (como es en el caso del precio de las opciones). Asumir que el proceso estocástico subyacente implica que las propiedades dependientes de los momentos se mantienen en el tiempo; en el caso de estacionaridad débil únicamente se requiere que el valor esperado y la función de covarianza no varíen con el tiempo y que exista el segundo momento.
\newline

\begin{figure}[hbt!]
    \centering
    \includegraphics[scale=0.5]{../images/series_temporales.png}
    \caption{Análisis de series temporales: Volatilidad histórica del S\&P 500 y volatilidad
  implícita (VIX)}
        \label{fig:series_temporales}   
\end{figure}

Si vemos el comportamiento del S\&P500 \ref{fig:series_temporales} vemos que no se cumplen condiciones de estacionaridad, en particular la volatilidad (la varianza) no se mantiene constante en el tiempo. ¿Será posible establecer un modelo para evaluar y estimar la heterocedasticidad de la serie de tiempo? Justamente esta pregunta es lo que buscan responder los modelos ARCH (''autoregressive conditional heteroscedasticity'') y GARCH (''general autoregressive conditional heteroscedasticity''). 
\newpage
\subsection{Modelo ARCH}
Sea $Z_t:= \ln\left(\frac{P_t}{P_{t-1}}\right)$ y sea $h_t= \mathbb{V}\left(Z_t|Z_s,s<t\right)$, el modelo ARCH plantea que: 
\begin{align*}
    Z_t = \sqrt{h_t} e_{t},
\end{align*}
donde $\{e_t\}$ es un proceso estocástico Normal$(0,1)$ independiente e idénticamente distribuido. En este caso, $P_t$, es el retorno en el día $t$ y $h_t$ es la volatilidad del modelo, esta va a estar relacionada con el proceso $\{Z_t^2\}$ por medio de la relación: 
\begin{align*}
    h_t = \alpha_0 + \sum_{i=1}^{p} \alpha_i Z_{t-i}^2,
\end{align*}
donde $\alpha_0>0$ y $\alpha_j\geq 0$ para $j\in \{1,\cdots,p\}$.
\subsubsection{Modelo ARCH(1)}

 Para el modelo ARCH(1) se tiene que:
\begin{align*}
    Z_t^2 &= h_t e_{t}^2\\
    &= e_t^2\left(\alpha_0 + \alpha_1 Z_{t-1}^2\right)\\
    &= e_t^2\left(\alpha_0 +  \alpha_1\alpha_0 e^{2}_{t} e^{2}_{t-1}+ \alpha_1^2 Z_{t-2}^2e^{ 2}_te^{2}_{t-1}\right)\\
    &= \cdots\\
    &= e_t^2\left(\alpha_0 \sum_{j=0}^n \alpha_1^{j}e_{t-1}^2\cdots e_{t-j}^2 +  \alpha_1^{n+1}Z_{t-n-1}^2  e^{2}_{t-1}\cdots e^{2}_{t-n}\right)\\
    &= \alpha_0 \sum_{j=0}^n \alpha_1^{j}e^{2}_t e_{t-1}^2\cdots e_{t-j}^2 +  \alpha_1^{n+1}Z_{t-n-1}^2  e^2_t e^{2}_{t-1}\cdots e^{2}_{t-n}.
\end{align*}
De esta última ecuación se tiene que si $\alpha_1<1$ y se asume que $\{Z_t\}$ es un proceso estacionario tal que $Z_t$ es una combinación lineal de $e_s$ para $s\leq t$, entonces:
\begin{align*}
    \mathbb{E}(\alpha_1^{n+1}Z_{t}^2)=\alpha_1^{n+1}\mathbb{E}(Z_{t}^2)\xrightarrow{\enskip n \enskip} 0 
\end{align*}  
Por lo tanto:
\begin{align*}
    \lim_{n\rightarrow \infty}\mathbb{E}(Z_t^2)= \alpha_0 \sum_{j=0}^\infty \alpha_1^{j}\mathbb{E}( e^{2}_t e_{t-1}^2\cdots e_{t-j}^2 )=\frac{\alpha_0}{1-\alpha_1}
\end{align*}
De este análisis anterior, se tiene la siguiente solución para el modelo ARCH(1):
Si $\alpha_1<1$ la solución causal del model ARCH(1) está dado por: 
\begin{align}
    Z_t=e^t\sqrt{\alpha_1\left(1+\sum_{j=1}^{\infty}\alpha_1^j e_{t-1}^2 \cdots e^{2}_{t-j} \right)},
\end{align}
y la solución tiene las siguientes propiedades:
\begin{align*}
    \mathbb{E}(Z_t)&=0\\
    \mathbb{V}(Z_t)&=\frac{\alpha_0}{1-\alpha_1}
\end{align*}
\subsubsection{ Modelo ARCH(p)}

Para el modelo ARCH(p) se tiene una solución similar a la del modelo ARCH(1). Se tiene que $Z_t^2$ es:

\begin{align*}
    Z_t^2 &= \left(\sqrt{h_t} e_t\right)^2 = h_t e_t^2\\
    &= \left(\alpha_0 + \sum_{i=1}^{p} \alpha_i Z_{t-i}^2\right) e_t^2\\
    &= \alpha_0 e_t^2 + \sum_{i=1}^{p} \alpha_i Z_{t-i}^2 e_t^2\\
    &= \alpha_0 e_t^2 + \sum_{i=1}^{p} \alpha_i e_t^2 e_{t-i}^2 \left(\alpha_0 + \sum_{j=1}^{p} \alpha_j Z_{t-i-j}^2\right)\\
    &= \alpha_0 e_t^2 + \alpha_0 \sum_{i=1}^{p} \alpha_i e_t^2 e_{t-i}^2 + \sum_{i=1}^{p} \sum_{j=1}^{p} \alpha_i \alpha_j e_t^2 e_{t-i}^2 Z_{t-i-j}^2\\
    &= \alpha_0 e_t^2 \left(1 + \sum_{k=1}^{\infty} \sum_{\mathbf{i} \in \mathcal{I}_k} \prod_{j=1}^{k} \alpha_{i_j} e_{t-s_j}^2\right) + \text{término residual},
\end{align*}

donde $\mathcal{I}_k$ denota el conjunto de secuencias de índices de longitud $k$ para $s_j$ tiempos correspondientes, y el término residual es el término que tiende a 0 cuando $n$ tiende a infinito. Mediante estos resultados se tiene que la solución para el modelo ARCH(p) cuando $\sum_{1}^p\alpha_j<1 $ y $\alpha_j\geq 0 $ para todo $j\in \{1,\cdots,p\}$ es: 
\begin{align*}
    Z_t^2= e^{t}Z_t=e^t\sqrt{\left(1 + \sum_{k=1}^{\infty} \sum_{\mathbf{i} \in \mathcal{I}_k} \prod_{j=1}^{k} \alpha_{i_j} e_{t-s_j}^2\right)}.
\end{align*}
\newline



\subsection{Modelo GARCH}

Para el modelo GARCH(p,q), la volatilidad condicional se extiende a:
\begin{align*}
    h_t = \alpha_0 + \sum_{i=1}^{p} \alpha_i Z_{t-i}^2 + \sum_{j=1}^{q} \beta_j h_{t-j}
\end{align*}

donde $\alpha_0 > 0$, $\alpha_i \geq 0$ para $i \in \{1, \ldots, p\}$, $\beta_j \geq 0$ para $j \in \{1, \ldots, q\}$, y para garantizar la estacionaridad en covarianza débil se requiere que $\sum_{i=1}^{\max(p,q)} (\alpha_i + \beta_i) < 1$.
\newline

El modelo GARCH generaliza el modelo ARCH permitiendo que la volatilidad condicional dependa no solo de los valores pasados de los retornos al cuadrado sino también de sus propios valores de la volatilidad del pasado. 
\subsubsection{Estimación por máxima verosimilitud}

Para el cálculo de los parámetros del modelo GARCH (al igual que para el modelo ARCH) se hace mediante la optimización de la función de verosimilitud. La función de verosimilitud para el modelo GARCH, dados los parámetros $\theta = (\mu, \alpha_0, \alpha_1, \ldots, \alpha_p, \beta_1, \ldots, \beta_q)$, asumiendo distribución normal condicional, es:
\begin{align*}
    L(\theta) = \prod_{t=1}^{T} \frac{1}{\sqrt{2\pi h_t}} \exp\left(-\frac{(Z_t - \mu)^2}{2h_t}\right)
\end{align*}

La log-verosimilitud correspondiente es:
\begin{align*}
    \ell(\theta) = -\frac{T}{2}\ln(2\pi) - \frac{1}{2}\sum_{t=1}^{T}\left[\ln(h_t) + \frac{(Z_t - \mu)^2}{h_t}\right]
\end{align*}

La estimación de máxima verosimilitud busca el vector $\hat{\theta}$ que maximiza $\ell(\theta)$:
\begin{align*}
    \hat{\theta} = \arg\max_{\theta \in \Theta} \ell(\theta)
\end{align*}

donde $\Theta$ es el espacio paramétrico que incluye las restricciones de no negatividad y estacionaridad.

\subsubsection{Solución analítica del modelo GARCH(1,1)}

El modelo GARCH(1,1) constituye la especificación más parsimoniosa y ampliamente utilizada en la literatura financiera para el modelado de volatilidad condicional. La ecuación de volatilidad se define como:
\begin{align}
    h_t = \omega + \alpha Z_{t-1}^2 + \beta h_{t-1}
\end{align}

donde $\omega = \alpha_0$, $\alpha = \alpha_1$ y $\beta = \beta_1$ representan los parámetros del término constante, del efecto ARCH y del efecto GARCH, respectivamente. La condición de estacionaridad requiere que $\alpha + \beta < 1$.

La sustitución recursiva hacia atrás permite derivar la representación de memoria infinita. Expandiendo la ecuación recursivamente:
\begin{align}
    h_t &= \omega + \alpha Z_{t-1}^2 + \beta h_{t-1}\\
    &= \omega + \alpha Z_{t-1}^2 + \beta(\omega + \alpha Z_{t-2}^2 + \beta h_{t-2})\\
    &= \omega(1 + \beta) + \alpha Z_{t-1}^2 + \alpha\beta Z_{t-2}^2 + \beta^2 h_{t-2}\\
    &= \omega \sum_{j=0}^{n-1} \beta^j + \alpha \sum_{j=0}^{n-1} \beta^j Z_{t-1-j}^2 + \beta^n h_{t-n}
\end{align}

Bajo la condición de estacionaridad y tomando el límite cuando $n \to \infty$, el término residual $\beta^n h_{t-n} \to 0$, obteniéndose:
\begin{align}
    h_t = \frac{\omega}{1-\beta} + \alpha \sum_{j=0}^{\infty} \beta^j Z_{t-1-j}^2
\end{align}

Esta expresión revela la naturaleza de la volatilidad condicional como una media ponderada exponencialmente decreciente de todos los choques pasados al cuadrado, donde el parámetro $\beta$ determina la velocidad de decaimiento. La constante $\frac{\omega}{1-\beta}$ representa el nivel base de volatilidad incondicional.

La varianza incondicional estacionaria se deriva aplicando el operador esperanza matemática, considerando que en el estado estacionario $\mathbb{E}[Z_t^2] = \sigma^2$ y $\mathbb{E}[h_t] = \sigma^2$:
\begin{align}
    \sigma^2 = \frac{\omega}{1 - \alpha - \beta}
\end{align}

Esta relación fundamental conecta los parámetros del modelo con la volatilidad incondicional de largo plazo.
\newline

El modelo GARCH(1,1) presenta características teóricas distintivas que lo hacen particularmente atractivo para aplicaciones financieras. La persistencia de volatilidad, medida por $\alpha + \beta$, captura la tendencia de los choques de volatilidad a tener efectos prolongados. Valores cercanos a la unidad indican alta persistencia, consistente con la evidencia empírica en mercados financieros donde los efectos de choques pueden persistir durante semanas o meses.
\newline

El fenómeno de clustering de volatilidad emerge naturalmente del modelo, donde períodos de alta volatilidad tienden a ser seguidos por períodos de alta volatilidad, y viceversa. Esta característica replica el comportamiento estilizado observado en series financieras, donde la volatilidad se agrupa temporalmente.

La vida media de un choque de volatilidad se calcula como $\frac{\ln(0.5)}{\ln(\alpha + \beta)}$, proporcionando una medida intuitiva del tiempo necesario para que el impacto de un choque se reduzca a la mitad. Esta métrica es particularmente relevante para la gestión de riesgo y la toma de decisiones de inversión.

\subsection{Volatilidad implícita y el índice VIX}

La volatilidad implícita constituye una medida prospectiva de la incertidumbre del mercado, derivada de los precios de opciones mediante la inversión del modelo de Black-Scholes. A diferencia de la volatilidad histórica, que se basa en observaciones pasadas, la volatilidad implícita incorpora las expectativas colectivas de los participantes del mercado sobre la volatilidad futura del activo subyacente.

\subsubsection{Fundamentos teóricos del VIX}

El VIX (Volatility Index), desarrollado por el Chicago Board Options Exchange (CBOE), representa el estándar de facto para medir la volatilidad implícita del mercado estadounidense. Su construcción se basa en una cartera sintética de opciones sobre el S\&P 500, proporcionando una medida model-free de la volatilidad esperada a 30 días.

La metodología del VIX emplea la fórmula de replicación de varianza desarrollada por Demeterfi et al. (1999), que aproxima la varianza futura mediante una integral de precios de opciones:
\begin{align}
    VIX = 100 \times \sqrt{\frac{2}{T} \sum_i \frac{\Delta K_i}{K_i^2} e^{rT} Q(K_i) - \frac{1}{T}\left[\frac{F}{K_0} - 1\right]^2}
\end{align}

En esta formulación, $T$ representa el tiempo hasta la expiración, $F$ el precio forward del S\&P 500, $K_i$ los strikes de las opciones, $\Delta K_i$ los intervalos entre strikes, $r$ la tasa libre de riesgo, $Q(K_i)$ el precio promedio bid-ask de cada opción, y $K_0$ el strike más cercano al precio forward.

\subsubsection{Comparación teórica: volatilidad histórica vs implícita}

En mercados eficientes, la volatilidad implícita debería constituir un predictor superior de la volatilidad futura realizada, dado que incorpora toda la información disponible al momento de la valoración. No obstante, la evidencia empírica documenta sistemáticamente la existencia de un "sesgo de volatilidad" , donde la volatilidad implícita excede consistentemente a la volatilidad realizada ex-post.
\newline

Este fenómeno sugiere que los inversionistas están dispuestos a pagar una prima por protección contra riesgo de volatilidad, reflejando aversión al riesgo y preferencias por seguros contra movimientos adversos del mercado. La magnitud y persistencia de esta prima de riesgo de volatilidad constituye un área activa de investigación en finanzas cuantitativas.

\subsection{Análisis empírico de volatilidad}

La evidencia empírica revela claramente la presencia de clusters de volatilidad, donde períodos de alta volatilidad se agrupan temporalmente, seguidos por períodos de relativa calma. Este patrón, inicialmente documentado por Mandelbrot (1963) y posteriormente formalizado por Engle (1982), constituye una de las principales motivaciones para el desarrollo de modelos ARCH/GARCH.
\newline

La heterocedasticidad condicional observada en los datos financieros refleja la naturaleza no constante de la varianza condicional a lo largo del tiempo. Esta característica contrasta fundamentalmente con los supuestos de homocedasticidad de los modelos clásicos de series temporales, justificando la necesidad de especificaciones más sofisticadas para capturar adecuadamente la dinámica de volatilidad.

Los modelos GARCH proporcionan un marco teórico robusto para modelar estos patrones, permitiendo que la volatilidad condicional evolucione dinámicamente en respuesta a nueva información. La capacidad de estos modelos para capturar tanto la persistencia como el clustering de volatilidad los convierte en herramientas fundamentales para aplicaciones en gestión de riesgo, pricing de derivados, y optimización de portafolios.

