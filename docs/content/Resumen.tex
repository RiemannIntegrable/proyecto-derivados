\section{Resumen}

Este estudio presenta una evaluación empírica comprehensiva del modelo GARCH(1,1) para la predicción de volatilidad en el mercado de acciones estadounidense, utilizando datos del índice S\&P 500 y estableciendo una comparación sistemática con la volatilidad implícita representada por el índice VIX. La investigación aborda una pregunta fundamental en finanzas cuantitativas: la capacidad relativa de modelos econométricos basados en información histórica versus medidas de volatilidad implícita derivadas de precios de derivados para predecir la volatilidad futura realizada.

La metodología empleada se fundamenta en técnicas econométricas robustas, implementando la estimación por máxima verosimilitud del modelo GARCH(1,1) con distribución normal condicional y utilizando algoritmos de optimización quasi-Newton con estimadores de covarianza robustos tipo Bollerslev-Wooldridge. Los datos comprenden series temporales de retornos logarítmicos del S\&P 500 obtenidos a través de la API de Yahoo Finance, procesados según las mejores prácticas en finanzas cuantitativas para asegurar la calidad y consistencia estadística de los resultados.

Los hallazgos principales revelan una superioridad estadística contundente del modelo GARCH sobre la volatilidad implícita del VIX en múltiples dimensiones de evaluación. Los criterios de información de Akaike y Bayesiano favorecen abrumadoramente al modelo GARCH con diferencias de 4418.08 y 4404.74 puntos respectivamente, constituyendo evidencia estadística irrefutable de su superior capacidad explicativa. Las métricas de precisión predictiva confirman esta superioridad, con el modelo GARCH exhibiendo un Error Porcentual Absoluto Medio (MAPE) de 49.57% comparado con 2424.08% para la volatilidad implícita, una diferencia que subraya la capacidad superior del enfoque econométrico para capturar la dinámica temporal de la volatilidad.

La estimación paramétrica revela características económicamente significativas del comportamiento de la volatilidad en mercados desarrollados. La persistencia de volatilidad extraordinariamente alta, medida por $\alpha_1 + \beta_1 = 0.9459$, indica que los choques de volatilidad en el mercado estadounidense poseen efectos extremadamente duraderos, con una vida media aproximada de 12 períodos. Esta persistencia excepcional es consistente con la evidencia empírica documentada sobre clustering de volatilidad en mercados financieros globales y tiene implicaciones profundas para la gestión de riesgo y el pricing de derivados.

El análisis de diagnóstico revela tanto las fortalezas como las limitaciones del marco teórico empleado. Mientras que el modelo GARCH demuestra efectividad superior en la captura de dependencia temporal y presenta menor evidencia de autocorrelación serial comparado con la volatilidad implícita, ambos enfoques exhiben desviaciones sistemáticas de la normalidad en los residuos, evidenciando la presencia de colas pesadas características de las series financieras. Esta violación del supuesto distribucional sugiere direcciones para extensiones futuras, incluyendo la incorporación de distribuciones de colas pesadas y la consideración de efectos de apalancamiento asimétricos.

Las implicaciones prácticas de estos resultados se extienden a múltiples áreas de las finanzas aplicadas. En gestión de riesgo, las estimaciones de volatilidad condicional GARCH pueden mejorar significativamente el cálculo de medidas como Value-at-Risk y Expected Shortfall, incorporando apropiadamente la dinámica temporal de la volatilidad. Para el pricing de derivados, el modelo ofrece estimaciones que pueden perfeccionar la aplicación de modelos fundamentales como Black-Scholes, especialmente relevante para la valuación de opciones de corto plazo donde la persistencia de volatilidad juega un papel crítico.

La superioridad empírica del modelo GARCH frente al VIX plantea cuestiones fundamentales sobre la eficiencia informativa de los mercados de derivados y sugiere la existencia de oportunidades para estrategias de inversión que combinen información econométrica con expectativas de mercado. Esta discrepancia entre volatilidad implícita y modelos históricos no necesariamente implica ineficiencia de mercado, sino que puede reflejar la incorporación de primas de riesgo, expectativas sobre eventos futuros, y dinámicas de demanda y oferta en el mercado de opciones que no son completamente capturadas por patrones históricos.

El estudio contribuye significativamente a la literatura de modelación de volatilidad al proporcionar evidencia empírica robusta sobre la efectividad comparativa de enfoques econométricos versus medidas de volatilidad implícita, utilizando criterios de evaluación comprehensivos y técnicas estadísticas rigurosas. Los resultados sugieren que, contrario a la intuición común sobre la superioridad de medidas forward-looking, los modelos GARCH pueden extraer información predictiva de patrones históricos que no es completamente incorporada en las expectativas de volatilidad implícita del mercado.

Las limitaciones identificadas incluyen la violación sistemática de supuestos distribucionales, la persistencia de heterocedasticidad residual, y la ausencia de efectos de apalancamiento en la especificación básica. Estas limitaciones sugieren direcciones prometedoras para investigación futura, incluyendo la implementación de modelos GARCH asimétricos (GJR-GARCH, EGARCH), la incorporación de distribuciones de colas pesadas (t-Student, GED), y la consideración de modelos de cambio de régimen para capturar adecuadamente episodios de crisis financieras.

En conclusión, este estudio demuestra que los modelos GARCH mantienen relevancia significativa en el arsenal de herramientas de finanzas cuantitativas, ofreciendo capacidad predictiva superior a medidas de volatilidad implícita ampliamente utilizadas. Los resultados subrayan la importancia de combinar enfoques econométricos rigurosos con comprensión profunda de la microestructura de mercados para desarrollar marcos de modelación de volatilidad más efectivos y económicamente significativos.