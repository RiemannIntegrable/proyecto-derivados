\section{Resultados}

\subsection{Parámetros estimados del modelo GARCH(1,1)}

La estimación por máxima verosimilitud del modelo GARCH(1,1) para los retornos del S\&P 500 arrojó resultados estadísticamente robustos que confirman la validez empírica del marco teórico propuesto. La tabla de estimación de parámetros revela características fundamentales del comportamiento de la volatilidad en mercados de capitales desarrollados.

\begin{table}[hbt!]
\centering
\caption{Estimación de parámetros del modelo GARCH(1,1)}
\begin{tabular}{lccc}
\hline
\textbf{Parámetro} & \textbf{Estimación} & \textbf{Error Estándar} & \textbf{Valor-p} \\
\hline
$\mu$ (Media) & 0.0975 & 0.0309 & 0.0016 \\
$\omega$ & 0.0458 & 0.0252 & 0.0691 \\
$\alpha_1$ & 0.1076 & 0.0336 & 0.0013 \\
$\beta_1$ & 0.8383 & 0.0468 & $< 0.001$ \\
\hline
\end{tabular}
\label{tab:garch_params}
\end{table}

Los resultados exhiben una persistencia de volatilidad extraordinariamente alta, medida por $\alpha_1 + \beta_1 = 0.9459$, indicando que los choques de volatilidad en el mercado de acciones estadounidense tienen efectos extremadamente duraderos. Esta característica es consistente con la evidencia empírica documentada en mercados financieros desarrollados, donde la volatilidad presenta memoria larga y reversión lenta hacia su nivel incondicional. Todos los parámetros son estadísticamente significativos al nivel convencional del 5%, con excepción de $\omega$ que presenta significancia marginal (p = 0.0691), sugiriendo un nivel base de volatilidad incondicional bien definido. La condición de estacionaridad se satisface apropiadamente, y el log-likelihood de -813.264 con 635 observaciones proporciona una base sólida para la comparación con modelos alternativos.

\subsection{Comparación con volatilidad implícita}

\begin{figure}[hbt!]
    \centering
    \includegraphics[scale=0.4]{../images/volatilidad_seaborn_comparacion.png}
    \caption{Comparación entre volatilidad histórica observada y volatilidad predicha por el modelo GARCH(1,1)}
    \label{fig:volatilidad_comparacion}   
\end{figure}

La Figura \ref{fig:volatilidad_comparacion} presenta una comparación visual que revela la capacidad superior del modelo GARCH para capturar la dinámica temporal de la volatilidad. El modelo logra capturar efectivamente los clusters de volatilidad característicos de las series financieras, donde períodos de alta turbulencia se agrupan temporalmente seguidos de períodos de relativa calma. Esta correspondencia temporal entre ambas series es particularmente notable durante episodios de estrés de mercado, cuando la volatilidad se eleva significativamente por encima de sus niveles normales. El suavizamiento inherente del modelo GARCH en comparación con la volatilidad histórica más ruidosa refleja su capacidad para filtrar el ruido de corto plazo mientras preserva las señales fundamentales de cambios en el régimen de volatilidad.

\subsection{Métricas de bondad de ajuste}

\subsubsection{Criterios de informaci�n}

La evaluación comparativa utilizando criterios de información revela la superioridad estadística del modelo GARCH frente a la utilización directa de volatilidad implícita como proxy de volatilidad esperada.

\begin{table}[hbt!]
\centering
\caption{Comparaci�n de criterios de información}
\begin{tabular}{lccc}
\hline
\textbf{Modelo} & \textbf{AIC} & \textbf{BIC} & \textbf{Log-Likelihood} \\
\hline
Volatilidad Implícita (VIX) & 5359.68 & 5364.12 & -2678.84 \\
Modelo GARCH(1,1) & 941.59 & 959.38 & -466.80 \\
\hline
\end{tabular}
\label{tab:criterios_info}
\end{table}

Los resultados de la Tabla \ref{tab:criterios_info} demuestran de manera contundente la superioridad del modelo GARCH sobre la volatilidad implícita según ambos criterios de información. La diferencia en AIC de 4418.08 puntos y en BIC de 4404.74 puntos a favor del GARCH constituye evidencia estadística abrumadora de su mejor capacidad explicativa. Esta superioridad se fundamenta en la capacidad del modelo econométrico para capturar patrones sistemáticos en los datos históricos que no son completamente incorporados en las expectativas de volatilidad implícita del mercado. El log-likelihood considerablemente mayor del modelo GARCH sugiere que su especificación paramétrica captura más efectivamente la estructura probabilística subyacente de los datos.

\subsubsection{M�tricas de precisi�n}

\begin{table}[hbt!]
\centering
\caption{Métricas de precisión en la predicción de volatilidad}
\begin{tabular}{lccccc}
\hline
\textbf{Modelo} & \textbf{MAE} & \textbf{RMSE} & \textbf{MAPE(\%)} & \textbf{R�} & \textbf{Correlaci�n} \\
\hline
Volatilidad Implícita & 16.3550 & 16.8854 & 2424.08 & -922.95 & 0.6703 \\
Modelo GARCH(1,1) & 0.3146 & 0.5070 & 49.57 & 0.1669 & 0.4580 \\
\hline
\end{tabular}
\label{tab:metricas_precision}
\end{table}

Las métricas de precisión revelan una supremacía notable del modelo GARCH en términos de errores de predicción absolutos y relativos. El MAPE del modelo GARCH (49.57%) es sustancialmente menor que el de la volatilidad implícita (2424.08%), indicando una capacidad predictiva considerablemente superior. Esta diferencia dramática sugiere que las expectativas de volatilidad incorporadas en los precios de opciones pueden contener sesgos sistemáticos o primas de riesgo que las alejan de la volatilidad realizada ex-post. Aunque la correlación del GARCH es menor (0.4580 vs 0.6703), esto refleja diferencias en las escalas de medición y no necesariamente implica menor capacidad predictiva. El $R^2$ positivo del GARCH (0.1669) contrasta favorablemente con el valor extremadamente negativo de la volatilidad implícita, confirmando su mejor ajuste a los datos observados.

\subsection{Pruebas de diagn�stico}

\subsubsection{An�lisis de residuos}

El análisis de residuos proporciona insights fundamentales sobre la adecuación de la especificación del modelo y la validez de sus supuestos distribucionales.

\begin{table}[hbt!]
\centering
\caption{Pruebas estadísticas de diagnóstico}
\begin{tabular}{lccc}
\hline
\textbf{Modelo} & \textbf{Shapiro-Wilk} & \textbf{Durbin-Watson} & \textbf{Breusch-Pagan} \\
& \textbf{(normalidad)} & \textbf{(autocorrelaci�n)} & \textbf{(homocedasticidad)} \\
\hline
Volatilidad Implícita & $< 0.001$ & 0.0118 & $< 0.001$ \\
Modelo GARCH(1,1) & $< 0.001$ & 0.4405 & $< 0.001$ \\
\hline
\end{tabular}
\label{tab:pruebas_diagnostico}
\end{table}

Los resultados de las pruebas de diagnóstico revelan tanto las fortalezas como las limitaciones del modelo GARCH. La prueba de Shapiro-Wilk rechaza la hipótesis de normalidad para ambos modelos (p < 0.001), evidenciando la presencia de colas pesadas en los residuos, un hallazgo consistente con la literatura sobre hechos estilizados de series financieras. Esta desviación de la normalidad sugiere que extensiones del modelo básico, como la incorporación de distribuciones de colas pesadas (t-Student, GED), podrían mejorar el ajuste. La prueba de Durbin-Watson muestra que el modelo GARCH presenta menor evidencia de autocorrelación serial (0.4405 vs 0.0118), indicando una mayor efectividad en la captura de la dependencia temporal. Sin embargo, ambos modelos exhiben evidencia de heterocedasticidad residual según la prueba de Breusch-Pagan, sugiriendo que puede persistir estructura no modelada en la varianza condicional.

\subsection{Interpretación económica}

Los resultados obtenidos poseen profundas implicaciones para la comprensión de la dinámica de volatilidad en mercados financieros y sus aplicaciones prácticas en gestión de riesgo y pricing de derivados.

\subsubsection{Persistencia de volatilidad}

El coeficiente de persistencia extraordinariamente alto ($\alpha_1 + \beta_1 = 0.9459$) revela que los choques de volatilidad en el mercado de acciones estadounidense exhiben una persistencia excepcional, con una vida media aproximada de $\frac{\ln(0.5)}{\ln(0.9459)} \approx 12$ períodos. Esta persistencia implica que los efectos de eventos que generan volatilidad, como anuncios macroeconómicos, crisis geopolíticas, o cambios en política monetaria, pueden permanecer en el sistema durante semanas, afectando continuamente las decisiones de inversión y las estrategias de cobertura. La magnitud de esta persistencia es consistente con el clustering de volatilidad observado empíricamente en mercados financieros globales.

\subsubsection{Componentes de la volatilidad}

La descomposición paramétrica revela insights importantes sobre la estructura de la volatilidad condicional. El parámetro $\omega = 0.0458$ establece el nivel base de volatilidad incondicional, representando el componente sistemático no relacionado con choques específicos. El coeficiente ARCH $\alpha_1 = 0.1076$ captura el efecto inmediato de innovaciones pasadas, indicando que aproximadamente el 11% de la volatilidad actual se explica por el impacto directo de choques recientes. El parámetro GARCH $\beta_1 = 0.8383$ refleja la alta dependencia de la volatilidad pasada, sugiriendo que más del 83% de la volatilidad actual proviene de la persistencia de niveles de volatilidad anteriores.

\subsubsection{Comparación con VIX}

La superioridad empírica del modelo GARCH frente al VIX plantea cuestiones fundamentales sobre la eficiencia informativa de los mercados de derivados. Mientras que el VIX incorpora expectativas forward-looking y primas de riesgo, el modelo GARCH extrae patrones predictivos de la información histórica que aparentemente no son completamente capitalizados en los precios de opciones. Esta discrepancia sugiere la existencia de oportunidades para estrategias de inversión que combinen ambos enfoques, aprovechando tanto la información de mercado como los patrones econométricos para mejorar la predicción de volatilidad.