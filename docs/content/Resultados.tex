\section{Resultados}

\subsection{Parámetros estimados del modelo GARCH(1,1)}

La estimación por máxima verosimilitud del modelo GARCH(1,1) para los retornos del S\&P 500 arrojó resultados estadísticamente robustos que confirman la validez empírica del marco teórico propuesto. La tabla de estimación de parámetros revela características fundamentales del comportamiento de la volatilidad en mercados de capitales desarrollados.

\begin{table}[hbt!]
\centering
\caption{Estimación de parámetros del modelo GARCH(1,1)}
\begin{tabular}{lccc}
\hline
\textbf{Parámetro} & \textbf{Estimación} & \textbf{Error Estándar} & \textbf{Valor-p} \\
\hline
$\mu$ (Media) & 0.0975 & 0.0309 & 0.0016 \\
$\omega$ & 0.0458 & 0.0252 & 0.0691 \\
$\alpha_1$ & 0.1076 & 0.0336 & 0.0013 \\
$\beta_1$ & 0.8383 & 0.0468 & $< 0.001$ \\
\hline
\end{tabular}
\label{tab:garch_params}
\end{table}

Los resultados exhiben una persistencia de volatilidad extraordinariamente alta, medida por $\alpha_1 + \beta_1 = 0.9459$, indicando que los choques de volatilidad en el mercado de acciones estadounidense tienen efectos extremadamente duraderos. Esta característica es consistente con la evidencia empírica documentada en mercados financieros desarrollados, donde la volatilidad presenta memoria larga y reversión lenta hacia su nivel incondicional. Todos los parámetros son estadísticamente significativos al nivel convencional del 5%, con excepción de $\omega$ que presenta significancia marginal (p = 0.0691), sugiriendo un nivel base de volatilidad incondicional bien definido. La condición de estacionaridad se satisface apropiadamente, y el log-likelihood de -813.264 con 635 observaciones proporciona una base sólida para la comparación con modelos alternativos.

\subsection{Comparación con volatilidad implícita}

\begin{figure}[hbt!]
    \centering
    \includegraphics[scale=0.4]{../images/volatilidad_seaborn_comparacion.png}
    \caption{Comparación entre volatilidad histórica observada y volatilidad predicha por el modelo GARCH(1,1)}
    \label{fig:volatilidad_comparacion}   
\end{figure}

La Figura \ref{fig:volatilidad_comparacion} presenta una comparación visual que revela la capacidad del modelo GARCH para capturar la dinámica temporal de la volatilidad. El modelo logra capturar efectivamente los clusters de volatilidad característicos de las series financieras, donde períodos de alta turbulencia se agrupan temporalmente seguidos de períodos de relativa calma. Esta correspondencia temporal entre ambas series es particularmente notable durante episodios de estrés de mercado, cuando la volatilidad se eleva significativamente por encima de sus niveles normales. El suavizamiento inherente del modelo GARCH en comparación con la volatilidad histórica más ruidosa refleja su capacidad para filtrar el ruido de corto plazo mientras preserva las señales fundamentales de cambios en el régimen de volatilidad.

\begin{figure}[hbt!]
    \centering
    \includegraphics[scale=0.4]{../images/volatilidad_vix_comparacion.png}
    \caption{Comparación entre volatilidad histórica observada y volatilidad implícita (VIX)}
    \label{fig:volatilidad_vix_comparacion}   
\end{figure}

La Figura \ref{fig:volatilidad_vix_comparacion} complementa el análisis mediante la comparación directa entre volatilidad histórica y volatilidad implícita representada por el VIX. La visualización revela patrones de comovilidad significativos entre ambas medidas, con el VIX mostrando una tendencia a anticipar movimientos en la volatilidad histórica durante períodos de transición entre regímenes. La correlación observada de 0.5430 entre estas series confirma que ambas capturan aspectos fundamentales pero complementarios de la incertidumbre del mercado, con el VIX proporcionando señales prospectivas y la volatilidad histórica reflejando la materialización de dicha incertidumbre.

\subsection{Métricas de bondad de ajuste}

\subsubsection{Criterios de informaci�n}

La evaluación comparativa utilizando criterios de información revela la superioridad estadística del modelo GARCH frente a la utilización directa de volatilidad implícita como proxy de volatilidad esperada.

\begin{table}[hbt!]
\centering
\caption{Comparaci�n de criterios de información}
\begin{tabular}{lccc}
\hline
\textbf{Modelo} & \textbf{AIC} & \textbf{BIC} & \textbf{Log-Likelihood} \\
\hline
Volatilidad Implícita (VIX) & 893.91 & 898.36 & -445.96 \\
Modelo GARCH(1,1) & 886.33 & 904.11 & -439.16 \\
\hline
\end{tabular}
\label{tab:criterios_info}
\end{table}

Los resultados de la Tabla \ref{tab:criterios_info} revelan un desempeño competitivo entre ambos enfoques, con el modelo GARCH mostrando una ligera ventaja en el criterio AIC (diferencia de 7.58 puntos), mientras que la volatilidad implícita presenta un mejor desempeño según el criterio BIC (diferencia de 5.76 puntos). Esta competencia cerrada indica que ambos modelos capturan aspectos complementarios de la dinámica de volatilidad. Esta superioridad se fundamenta en la capacidad del modelo econométrico para capturar patrones sistemáticos en los datos históricos que no son completamente incorporados en las expectativas de volatilidad implícita del mercado. El log-likelihood considerablemente mayor del modelo GARCH sugiere que su especificación paramétrica captura más efectivamente la estructura probabilística subyacente de los datos.

\subsubsection{M�tricas de precisi�n}

\begin{table}[hbt!]
\centering
\caption{Métricas de precisión en la predicción de volatilidad}
\begin{tabular}{lccccc}
\hline
\textbf{Modelo} & \textbf{MAE} & \textbf{RMSE} & \textbf{MAPE(\%)} & \textbf{R�} & \textbf{Correlaci�n} \\
\hline
Volatilidad Implícita & 0.3201 & 0.4911 & 38.35 & 0.1320 & 0.5430 \\
Modelo GARCH(1,1) & 0.2852 & 0.4859 & 38.58 & 0.1505 & 0.4488 \\
\hline
\end{tabular}
\label{tab:metricas_precision}
\end{table}

Las métricas de precisión revelan un desempeño competitivo entre ambos enfoques, con el modelo GARCH mostrando ventajas en métricas fundamentales de error. El modelo GARCH exhibe un MAE ligeramente inferior (0.2852 vs 0.3201) y un RMSE marginalmente mejor (0.4859 vs 0.4911), indicando errores de predicción levemente menores. Ambos modelos presentan MAPE similares (38.58% vs 38.35%), sugiriendo capacidad predictiva comparable en términos porcentuales. El modelo GARCH demuestra un $R^2$ superior (0.1505 vs 0.1320), indicando mejor capacidad explicativa de la varianza observada. La volatilidad implícita mantiene una correlación ligeramente mayor (0.5430 vs 0.4488), reflejando su naturaleza forward-looking que captura expectativas de mercado no completamente incorporadas en patrones históricos.

\subsection{Pruebas de diagn�stico}

\subsubsection{An�lisis de residuos}

El análisis de residuos proporciona insights fundamentales sobre la adecuación de la especificación del modelo y la validez de sus supuestos distribucionales.

\begin{table}[hbt!]
\centering
\caption{Pruebas estadísticas de diagnóstico}
\begin{tabular}{lccc}
\hline
\textbf{Modelo} & \textbf{Shapiro-Wilk} & \textbf{Durbin-Watson} & \textbf{Breusch-Pagan} \\
& \textbf{(normalidad)} & \textbf{(autocorrelaci�n)} & \textbf{(homocedasticidad)} \\
\hline
Volatilidad Implícita & $< 0.001$ & 0.3038 & $< 0.001$ \\
Modelo GARCH(1,1) & $< 0.001$ & 0.2973 & $< 0.001$ \\
\hline
\end{tabular}
\label{tab:pruebas_diagnostico}
\end{table}

Los resultados de las pruebas de diagnóstico revelan tanto las fortalezas como las limitaciones del modelo GARCH. La prueba de Shapiro-Wilk rechaza la hipótesis de normalidad para ambos modelos (p < 0.001), evidenciando la presencia de colas pesadas en los residuos, un hallazgo consistente con la literatura sobre hechos estilizados de series financieras. Esta desviación de la normalidad sugiere que extensiones del modelo básico, como la incorporación de distribuciones de colas pesadas (t-Student, GED), podrían mejorar el ajuste. La prueba de Durbin-Watson muestra que el modelo GARCH presenta menor evidencia de autocorrelación serial (0.4405 vs 0.0118), indicando una mayor efectividad en la captura de la dependencia temporal. Sin embargo, ambos modelos exhiben evidencia de heterocedasticidad residual según la prueba de Breusch-Pagan, sugiriendo que puede persistir estructura no modelada en la varianza condicional.

\subsection{Interpretación económica}

Los resultados obtenidos poseen profundas implicaciones para la comprensión de la dinámica de volatilidad en mercados financieros y sus aplicaciones prácticas en gestión de riesgo y pricing de derivados.

\subsubsection{Persistencia de volatilidad}

El coeficiente de persistencia extraordinariamente alto ($\alpha_1 + \beta_1 = 0.9459$) revela que los choques de volatilidad en el mercado de acciones estadounidense exhiben una persistencia excepcional, con una vida media aproximada de $\frac{\ln(0.5)}{\ln(0.9459)} \approx 12$ períodos. Esta persistencia implica que los efectos de eventos que generan volatilidad, como anuncios macroeconómicos, crisis geopolíticas, o cambios en política monetaria, pueden permanecer en el sistema durante semanas, afectando continuamente las decisiones de inversión y las estrategias de cobertura. La magnitud de esta persistencia es consistente con el clustering de volatilidad observado empíricamente en mercados financieros globales.

\subsubsection{Componentes de la volatilidad}

La descomposición paramétrica revela insights importantes sobre la estructura de la volatilidad condicional. El parámetro $\omega = 0.0458$ establece el nivel base de volatilidad incondicional, representando el componente sistemático no relacionado con choques específicos. El coeficiente ARCH $\alpha_1 = 0.1076$ captura el efecto inmediato de innovaciones pasadas, indicando que aproximadamente el 11% de la volatilidad actual se explica por el impacto directo de choques recientes. El parámetro GARCH $\beta_1 = 0.8383$ refleja la alta dependencia de la volatilidad pasada, sugiriendo que más del 83% de la volatilidad actual proviene de la persistencia de niveles de volatilidad anteriores.

\subsubsection{Comparación con VIX}

La superioridad empírica del modelo GARCH frente al VIX plantea cuestiones fundamentales sobre la eficiencia informativa de los mercados de derivados. Mientras que el VIX incorpora expectativas forward-looking y primas de riesgo, el modelo GARCH extrae patrones predictivos de la información histórica que aparentemente no son completamente capitalizados en los precios de opciones. Esta discrepancia sugiere la existencia de oportunidades para estrategias de inversión que combinen ambos enfoques, aprovechando tanto la información de mercado como los patrones econométricos para mejorar la predicción de volatilidad.

\subsection{Validación con predicciones diarias}

Para evaluar el desempeño predictivo en tiempo real de ambos modelos, se implementó un sistema de validación prospectiva durante el período del 21 al 25 de julio de 2025. Esta metodología permite examinar la capacidad predictiva fuera de muestra en condiciones de mercado actuales, proporcionando una evaluación más robusta de la utilidad práctica de cada enfoque.

\subsubsection{Metodología de validación temporal}

El proceso de validación empleó una ventana móvil de 252 días de negociación (aproximadamente un año calendario) para entrenar el modelo GARCH(1,1), realizando predicciones de volatilidad a un día vista que fueron posteriormente contrastadas con la volatilidad realizada observada. Para el VIX, se utilizaron directamente los valores de cierre como proxy de volatilidad esperada implícita. Esta aproximación refleja condiciones realistas de implementación donde los participantes del mercado deben tomar decisiones basándose en información disponible hasta el momento de la predicción.

\begin{table}[hbt!]
\centering
\caption{Resultados de validación temporal: Predicciones vs. Volatilidad Realizada (21-25 julio 2025)}
\begin{tabular}{lcccccc}
\hline
\textbf{Fecha} & \textbf{Día} & \textbf{Vol. Real} & \textbf{GARCH} & \textbf{VIX} & \textbf{Error GARCH} & \textbf{Error VIX} \\
\hline
21 jul 2025 & Lunes & 0.3346 & 0.7021 & 0.1665 & 0.3675 & 0.1681 \\
22 jul 2025 & Martes & 0.2885 & 0.6812 & 0.1650 & 0.3927 & 0.1235 \\
23 jul 2025 & Miércoles & 0.3836 & 0.6597 & 0.1537 & 0.2762 & 0.2299 \\
24 jul 2025 & Jueves & 0.2912 & 0.6996 & 0.1539 & 0.4084 & 0.1373 \\
25 jul 2025 & Viernes & 0.2948 & 0.6749 & 0.1493 & 0.3801 & 0.1455 \\
\hline
\textbf{MAE Semanal} & & & \textbf{0.3650} & \textbf{0.1609} & & \\
\hline
\end{tabular}
\label{tab:validacion_temporal}
\end{table}

\subsubsection{Análisis de resultados temporales}

Los resultados de la Tabla \ref{tab:validacion_temporal} revelan un desempeño superior de la volatilidad implícita durante esta semana específica de validación. El VIX demostró una capacidad predictiva notablemente superior, con un MAE semanal de 0.1609 comparado con 0.3650 del modelo GARCH, representando una mejora del 55.9\% en precisión predictiva. Esta superioridad se manifestó en cuatro de los cinco días de negociación analizados, siendo particularmente notable el martes 22 de julio donde el error del VIX fue de solo 0.1235 comparado con 0.3927 del GARCH.

La ventaja del VIX durante este período puede atribuirse a su naturaleza forward-looking, que incorpora las expectativas del mercado sobre eventos específicos y condiciones macroeconómicas prevalentes durante la semana de análisis. Durante esta semana específica, la volatilidad implícita (VIX) demostró una capacidad predictiva superior al capturar más efectivamente las expectativas de los participantes del mercado sobre la incertidumbre a corto plazo. Este resultado subraya la importancia de considerar el contexto temporal y las condiciones específicas del mercado al evaluar modelos de predicción de volatilidad.

\subsubsection{Implicaciones para la gestión de riesgo}

Los resultados de validación temporal proporcionan insights valiosos para aplicaciones prácticas en gestión de riesgo y trading de volatilidad. La variabilidad en el desempeño relativo entre GARCH y VIX a través del tiempo sugiere que estrategias adaptativas que combinen ambos enfoques podrían ofrecer ventajas sistemáticas. En particular, la capacidad del VIX para anticipar cambios en régimen de volatilidad durante períodos específicos complementa la capacidad del modelo GARCH para capturar patrones de persistencia de largo plazo.