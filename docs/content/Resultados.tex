\section{Resultados}

\subsection{Parámetros estimados del modelo GARCH(1,1)}

La estimación por máxima verosimilitud del modelo GARCH(1,1) para los retornos del S\&P 500 arrojó los siguientes resultados:

\begin{table}[hbt!]
\centering
\caption{Estimación de parámetros del modelo GARCH(1,1)}
\begin{tabular}{lccc}
\hline
\textbf{Parámetro} & \textbf{Estimación} & \textbf{Error Estándar} & \textbf{Valor-p} \\
\hline
$\mu$ (Media) & 0.0975 & 0.0309 & 0.0016 \\
$\omega$ & 0.0458 & 0.0252 & 0.0691 \\
$\alpha_1$ & 0.1076 & 0.0336 & 0.0013 \\
$\beta_1$ & 0.8383 & 0.0468 & $< 0.001$ \\
\hline
\end{tabular}
\label{tab:garch_params}
\end{table}

Los resultados muestran que:
\begin{itemize}
    \item La persistencia de la volatilidad ($\alpha_1 + \beta_1 = 0.9459$) es alta, indicando que los choques de volatilidad tienen efectos duraderos
    \item Todos los parámetros son significativos al nivel del 5\% excepto $\omega$, que es marginalmente significativo (p = 0.0691)
    \item La condición de estacionaridad se cumple ($\alpha_1 + \beta_1 < 1$)
    \item El log-likelihood del modelo es -813.264 con 635 observaciones
\end{itemize}

\subsection{Comparación con volatilidad implícita}

\begin{figure}[hbt!]
    \centering
    \includegraphics[scale=0.4]{../images/volatilidad_seaborn_comparacion.png}
    \caption{Comparación entre volatilidad histórica observada y volatilidad predicha por el modelo GARCH(1,1)}
    \label{fig:volatilidad_comparacion}   
\end{figure}

La Figura \ref{fig:volatilidad_comparacion} presenta una comparación visual entre la volatilidad histórica observada y la volatilidad predicha por el modelo GARCH(1,1). Se observa que:

\begin{itemize}
    \item El modelo GARCH captura efectivamente los clusters de volatilidad
    \item Existe una correspondencia razonable entre ambas series, especialmente durante períodos de alta volatilidad
    \item El modelo muestra cierto suavizamiento en comparación con la volatilidad histórica más ruidosa
\end{itemize}

\subsection{Métricas de bondad de ajuste}

\subsubsection{Criterios de informaci�n}

Se compararon dos enfoques para modelar la volatilidad utilizando criterios de información:

\begin{table}[hbt!]
\centering
\caption{Comparaci�n de criterios de información}
\begin{tabular}{lccc}
\hline
\textbf{Modelo} & \textbf{AIC} & \textbf{BIC} & \textbf{Log-Likelihood} \\
\hline
Volatilidad Implícita (VIX) & 5359.68 & 5364.12 & -2678.84 \\
Modelo GARCH(1,1) & 941.59 & 959.38 & -466.80 \\
\hline
\end{tabular}
\label{tab:criterios_info}
\end{table}

Los resultados de la Tabla \ref{tab:criterios_info} muestran que el modelo GARCH(1,1) supera significativamente a la volatilidad implícita según ambos criterios:
\begin{itemize}
    \item Diferencia en AIC: 4418.08 puntos a favor del GARCH
    \item Diferencia en BIC: 4404.74 puntos a favor del GARCH
    \item El modelo GARCH presenta un log-likelihood considerablemente mayor
\end{itemize}

\subsubsection{M�tricas de precisi�n}

\begin{table}[hbt!]
\centering
\caption{Métricas de precisión en la predicción de volatilidad}
\begin{tabular}{lccccc}
\hline
\textbf{Modelo} & \textbf{MAE} & \textbf{RMSE} & \textbf{MAPE(\%)} & \textbf{R�} & \textbf{Correlaci�n} \\
\hline
Volatilidad Implícita & 16.3550 & 16.8854 & 2424.08 & -922.95 & 0.6703 \\
Modelo GARCH(1,1) & 0.3146 & 0.5070 & 49.57 & 0.1669 & 0.4580 \\
\hline
\end{tabular}
\label{tab:metricas_precision}
\end{table}

Los resultados de la Tabla \ref{tab:metricas_precision} revelan:

\begin{itemize}
    \item El modelo GARCH muestra errores significativamente menores en todas las métricas de precisión
    \item El MAPE del GARCH (49.57\%) es considerablemente menor que el de la volatilidad implícita (2424.08\%)
    \item Aunque la correlación del GARCH es menor (0.4580 vs 0.6703), esto se debe a la diferencia en escalas de medición
    \item El $R^2$ del GARCH (0.1669) indica un ajuste modesto pero positivo
\end{itemize}

\subsection{Pruebas de diagn�stico}

\subsubsection{An�lisis de residuos}

Se aplicaron tres pruebas estadísticas principales para evaluar los supuestos del modelo:

\begin{table}[hbt!]
\centering
\caption{Pruebas estadísticas de diagnóstico}
\begin{tabular}{lccc}
\hline
\textbf{Modelo} & \textbf{Shapiro-Wilk} & \textbf{Durbin-Watson} & \textbf{Breusch-Pagan} \\
& \textbf{(normalidad)} & \textbf{(autocorrelaci�n)} & \textbf{(homocedasticidad)} \\
\hline
Volatilidad Implícita & $< 0.001$ & 0.0118 & $< 0.001$ \\
Modelo GARCH(1,1) & $< 0.001$ & 0.4405 & $< 0.001$ \\
\hline
\end{tabular}
\label{tab:pruebas_diagnostico}
\end{table}

Interpretación de los resultados:

\begin{itemize}
    \item \textbf{Shapiro-Wilk}: Ambos modelos rechazan la hipótesis de normalidad (p $< 0.001$), indicando colas pesadas en los residuos
    \item \textbf{Durbin-Watson}: El modelo GARCH muestra menor evidencia de autocorrelación serial (0.4405 vs 0.0118)
    \item \textbf{Breusch-Pagan}: Ambos modelos presentan evidencia de heterocedasticidad residual
\end{itemize}

\subsection{Interpretación económica}

Los resultados obtenidos tienen importantes implicaciones económicas y financieras:

\subsubsection{Persistencia de volatilidad}

El coeficiente de persistencia ($\alpha_1 + \beta_1 = 0.9459$) indica que:
\begin{itemize}
    \item Los choques de volatilidad en el mercado de acciones tienen efectos duraderos
    \item La vida media de un choque de volatilidad es aproximadamente $\frac{\ln(0.5)}{\ln(0.9459)} \approx 12$ períodos
    \item Esta persistencia es consistente con el clustering de volatilidad observado en mercados financieros
\end{itemize}

\subsubsection{Componentes de la volatilidad}

La descomposición de la ecuación de volatilidad revela:
\begin{itemize}
    \item $\omega = 0.0458$ representa el nivel base de volatilidad
    \item $\alpha_1 = 0.1076$ captura el efecto inmediato de choques pasados (efecto ARCH)
    \item $\beta_1 = 0.8383$ refleja la dependencia de la volatilidad pasada (efecto GARCH)
\end{itemize}

\subsubsection{Comparación con VIX}

El superior desempeño del modelo GARCH frente al VIX sugiere:
\begin{itemize}
    \item Los modelos econométricos pueden complementar la información de mercado
    \item El VIX refleja expectativas forward-looking, mientras que GARCH modela patrones históricos
    \item La combinación de ambos enfoques podría mejorar la predicción de volatilidad
\end{itemize}

\subsection{Aplicaciones prácticas}

Los resultados del modelo GARCH tienen múltiples aplicaciones:

\begin{enumerate}
    \item \textbf{Gestión de riesgo}: Estimación de Value-at-Risk (VaR) y Expected Shortfall
    \item \textbf{Pricing de derivados}: Mejora en la estimación de volatilidad para modelos de Black-Scholes
    \item \textbf{Optimización de portafolios}: Incorporación de volatilidad condicional en matrices de covarianza
    \item \textbf{Trading algorítmico}: Señales basadas en regímenes de volatilidad
\end{enumerate}

\subsection{Limitaciones identificadas}

A pesar de los buenos resultados, se identificaron las siguientes limitaciones:

\begin{itemize}
    \item Violación del supuesto de normalidad en los residuos
    \item Presencia de heterocedasticidad residual
    \item El modelo no captura completamente los efectos de apalancamiento (leverage effects)
    \item Limitaciones en la predicción durante eventos extremos del mercado
\end{itemize}

