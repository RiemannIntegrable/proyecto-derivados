\section{Implementación}

\begin{frame}{Configuración del Ambiente}
    \begin{columns}
        \begin{column}{0.6\textwidth}
            \begin{block}{Datos}
                \begin{itemize}
                    \item \textbf{Serie}: S\&P 500 (Yahoo Finance)
                    \item \textbf{VIX}: Volatilidad implícita
                    \item \textbf{Retornos}: $Z_t = \ln\left(\frac{P_t}{P_{t-1}}\right) \times 100$
                    \item \textbf{Volatilidad histórica}: ventana móvil 7 días
                \end{itemize}
            \end{block}
        \end{column}
        
        \begin{column}{0.4\textwidth}
            \begin{block}{Software}
                \begin{itemize}
                    \item \textbf{Python}
                    \item Librerías:
                    \begin{itemize}
                        \item \texttt{arch}
                        \item \texttt{statsmodels}
                        \item \texttt{yfinance}
                        \item \texttt{pandas}
                    \end{itemize}
                \end{itemize}
            \end{block}
        \end{column}
    \end{columns}
\end{frame}

\begin{frame}
    \begin{figure}
        \centering
        \includegraphics[width=0.8\textwidth]{../images/acf_y_pacf.png}
        \caption{ACF y PACF para retornos y volatilidad histórica del S\&P 500}
    \end{figure}
\end{frame}

\begin{frame}{Funciones de Autocorrelación}
    \begin{figure}
        \centering
        \includegraphics[width=0.5\textwidth]{../images/acf_y_pacf.png}
        \caption{ACF y PACF para retornos y volatilidad histórica del S\&P 500}
    \end{figure}
    
    \vspace{0.5em}
    
    \begin{itemize}
        \item<2-> \textbf{Retornos}: autocorrelación prácticamente nula
        \item<3-> \textbf{Volatilidad}: autocorrelación significativa y persistente
    \end{itemize}
\end{frame}

\begin{frame}[fragile]{Especificación del Modelo}
    \begin{block}{Código Python}
        \begin{verbatim}
modelo_garch = arch_model(
    retornos.dropna(),
    vol='GARCH',
    p=1,  # Orden ARCH
    q=1,  # Orden GARCH
    mean='Constant',
    dist='Normal',
    rescale=True
)
        \end{verbatim}
    \end{block}
    
    \vspace{0.5em}
    
    \begin{itemize}
        \item<2-> \textbf{Media constante}: $\mu$
        \item<3-> \textbf{Volatilidad GARCH(1,1)}: $h_t = \omega + \alpha_1 Z_{t-1}^2 + \beta_1 h_{t-1}$
        \item<4-> \textbf{Distribución normal} para residuos estandarizados
    \end{itemize}
\end{frame}

\begin{frame}{Estimación por Máxima Verosimilitud}
    \begin{block}{Función Log-Verosimilitud}
        \begin{align}
            \ell(\theta) = -\frac{T}{2}\ln(2\pi) - \frac{1}{2}\sum_{t=1}^{T}\left[\ln(h_t) + \frac{Z_t^2}{h_t}\right]
        \end{align}
        donde $\theta = (\mu, \omega, \alpha_1, \beta_1)$
    \end{block}
    
    \vspace{0.5em}
    
    \begin{itemize}
        \item<2-> \textbf{Optimización}: algoritmos robustos
        \item<3-> \textbf{Restricciones}: $\omega, \alpha_1, \beta_1 \geq 0$
        \item<4-> \textbf{Estacionaridad}: $\alpha_1 + \beta_1 < 1$
    \end{itemize}
\end{frame}

\begin{frame}{Métricas de Evaluación}
    \begin{enumerate}
        \item<1-> \textbf{Criterios de Información}
        \begin{align}
            AIC &= 2k - 2\ell(\hat{\theta})\\
            BIC &= k\ln(T) - 2\ell(\hat{\theta})
        \end{align}
        
        \item<2-> \textbf{Métricas de Precisión}
        \begin{align}
            MAE &= \frac{1}{T}\sum_{t=1}^{T}|\sigma_{t} - \hat{\sigma}_t|\\
            RMSE &= \sqrt{\frac{1}{T}\sum_{t=1}^{T}(\sigma_{t} - \hat{\sigma}_t)^2}
        \end{align}
        
        \item<3-> \textbf{Pruebas de Diagnóstico}
        \begin{itemize}
            \item Shapiro-Wilk (normalidad)
            \item Durbin-Watson (autocorrelación)
            \item Breusch-Pagan (homocedasticidad)
        \end{itemize}
    \end{enumerate}
\end{frame}