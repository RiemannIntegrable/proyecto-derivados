\section{Metodología}

\begin{frame}{Datos y Metodología}
    \begin{columns}
        \begin{column}{0.6\textwidth}
            \begin{block}{Datos}
                \begin{itemize}
                    \item Serie de tiempo: Índice COLCAP
                    \item Período: 2010-2024
                    \item Frecuencia: Diaria
                    \item Rendimientos logarítmicos: $r_t = \ln(P_t/P_{t-1})$
                \end{itemize}
            \end{block}
        \end{column}
        
        \begin{column}{0.4\textwidth}
            \begin{block}{Software}
                \begin{itemize}
                    \item \textbf{R} 4.4.0
                    \item Paquetes:
                    \begin{itemize}
                        \item \texttt{rugarch}
                        \item \texttt{rmgarch}
                        \item \texttt{quantmod}
                    \end{itemize}
                \end{itemize}
            \end{block}
        \end{column}
    \end{columns}
\end{frame}

\begin{frame}{Proceso de Modelado}
    \begin{enumerate}
        \item<1-> \textbf{Análisis Exploratorio}
        \begin{itemize}
            \item Estadísticas descriptivas
            \item Pruebas de raíz unitaria
            \item Análisis de autocorrelación
        \end{itemize}
        
        \item<2-> \textbf{Modelado de la Media}
        \begin{itemize}
            \item Selección de modelo ARMA para $\mu_t$
            \item Criterios de información (AIC, BIC)
        \end{itemize}
        
        \item<3-> \textbf{Pruebas de Heterocedasticidad}
        \begin{itemize}
            \item Test de Engle (ARCH-LM)
            \item Test de Ljung-Box en residuos al cuadrado
        \end{itemize}
        
        \item<4-> \textbf{Estimación GARCH}
        \begin{itemize}
            \item Máxima verosimilitud
            \item Comparación de especificaciones
        \end{itemize}
    \end{enumerate}
\end{frame}

\begin{frame}[fragile]{Implementación en R}
    \begin{lstlisting}[language=R]
# Especificación del modelo
spec <- ugarchspec(
    variance.model = list(
        model = "sGARCH",
        garchOrder = c(1, 1)
    ),
    mean.model = list(
        armaOrder = c(1, 0)
    ),
    distribution.model = "norm"
)

# Estimación
fit <- ugarchfit(spec, data = returns)
    \end{lstlisting}
\end{frame}