\section{Resultados}

\begin{frame}{Parámetros Estimados GARCH(1,1)}
    \begin{table}
        \centering
        \begin{tabular}{lccc}
            \toprule
            \textbf{Parámetro} & \textbf{Estimación} & \textbf{Error Estándar} & \textbf{Valor-p} \\
            \midrule
            $\mu$ (Media) & 0.0975 & 0.0309 & 0.0016 \\
            $\omega$ & 0.0458 & 0.0252 & 0.0691 \\
            $\alpha_1$ & 0.1076 & 0.0336 & 0.0013 \\
            $\beta_1$ & 0.8383 & 0.0468 & $< 0.001$ \\
            \bottomrule
        \end{tabular}
        \caption{Parámetros del modelo GARCH(1,1) para S\&P 500}
    \end{table}
    
    \vspace{0.5em}
    
    \begin{itemize}
        \item<2-> \highlight{Alta persistencia}: $\alpha_1 + \beta_1 = 0.9459$
        \item<3-> Parámetros significativos al 5\% (excepto $\omega$ marginalmente)
        \item<4-> Condición de estacionaridad se cumple
        \item<5-> Log-likelihood: -813.264 con 635 observaciones
    \end{itemize}
\end{frame}

\begin{frame}{Interpretación Económica}
    \begin{block}{Persistencia de Volatilidad}
        Coeficiente $\alpha_1 + \beta_1 = 0.9459$ indica:
        \begin{itemize}
            \item Choques de volatilidad tienen \emphasis{efectos duraderos}
            \item Vida media $\approx \frac{\ln(0.5)}{\ln(0.9459)} \approx 12$ períodos
            \item Consistente con clustering de volatilidad
        \end{itemize}
    \end{block}
    
    \vspace{0.5em}
    
    \begin{block}{Componentes de Volatilidad}
        \begin{itemize}
            \item<2-> $\omega = 0.0458$: nivel base de volatilidad
            \item<3-> $\alpha_1 = 0.1076$: efecto inmediato (ARCH)
            \item<4-> $\beta_1 = 0.8383$: dependencia del pasado (GARCH)
        \end{itemize}
    \end{block}
\end{frame}

\begin{frame}{Comparación Visual: GARCH vs Volatilidad Histórica}
    \begin{figure}
        \centering
        \includegraphics[width=0.7\textwidth]{../images/volatilidad_seaborn_comparacion.png}
        \caption{Volatilidad histórica vs predicción GARCH(1,1)}
    \end{figure}
        
    \begin{itemize}
        \item<2-> GARCH captura efectivamente los \emphasis{clusters de volatilidad}
        \item<3-> Correspondencia razonable durante períodos de alta volatilidad
    \end{itemize}
\end{frame}

\begin{frame}{Criterios de Información}
    \begin{table}
        \centering
        \begin{tabular}{lccc}
            \toprule
            \textbf{Modelo} & \textbf{AIC} & \textbf{BIC} & \textbf{Log-Likelihood} \\
            \midrule
            Volatilidad Implícita (VIX) & 5359.68 & 5364.12 & -2678.84 \\
            Modelo GARCH(1,1) & 941.59 & 959.38 & -466.80 \\
            \bottomrule
        \end{tabular}
        \caption{Comparación de criterios de información}
    \end{table}
    
    \vspace{0.5em}
    
    \begin{itemize}
        \item<2-> \highlight{GARCH supera significativamente al VIX}:
        \begin{itemize}
            \item Diferencia AIC: 4418.08 puntos
            \item Diferencia BIC: 4404.74 puntos
        \end{itemize}
        \item<3-> Log-likelihood considerablemente mayor para GARCH
    \end{itemize}
\end{frame}

\begin{frame}{Fórmulas de Evaluación}
    \begin{itemize}
        \item \textbf{MAE}:
        {\footnotesize $$MAE = \frac{1}{n}\sum_{i=1}^{n}|y_i - \hat{y}_i|$$}
  
        \item \textbf{RMSE}:
        {\footnotesize $$RMSE = \sqrt{\frac{1}{n}\sum_{i=1}^{n}(y_i - \hat{y}_i)^2}$$}
  
        \item \textbf{MAPE}:
        {\footnotesize $$MAPE = \frac{100}{n}\sum_{i=1}^{n}\left|\frac{y_i - \hat{y}_i}{y_i}\right|$$}
  
        \item \textbf{R²}:
        {\footnotesize $$R^2 = 1 - \frac{SS_{res}}{SS_{tot}} = 1 - \frac{\sum_{i=1}^{n}(y_i - \hat{y}_i)^2}{\sum_{i=1}^{n}(y_i - \bar{y})^2}$$}
  
        \item \textbf{Correlación}:
        {\footnotesize $$r = \frac{\sum_{i=1}^{n}(x_i - \bar{x})(y_i - \bar{y})}{\sqrt{\sum_{i=1}^{n}(x_i - \bar{x})^2 \sum_{i=1}^{n}(y_i - \bar{y})^2}}$$}
    \end{itemize}
\end{frame}

\begin{frame}{Métricas de Precisión}
    \begin{table}
        \centering
        \footnotesize
        \begin{tabular}{lccccc}
            \toprule
            \textbf{Modelo} & \textbf{MAE} & \textbf{RMSE} & \textbf{MAPE(\%)} & \textbf{R²} & \textbf{Correlación} \\
            \midrule
            Volatilidad Implícita & 16.36 & 16.89 & 2424.08 & -922.95 & 0.6703 \\
            Modelo GARCH(1,1) & 0.31 & 0.51 & 49.57 & 0.1669 & 0.4580 \\
            \bottomrule
        \end{tabular}
        \caption{Métricas de precisión en predicción de volatilidad}
    \end{table}
    
    \vspace{0.5em}
    
    \begin{itemize}
        \item<2-> GARCH muestra \highlight{errores significativamente menores}
        \item<3-> MAPE del GARCH (49.57\%) << VIX (2424.08\%)
        \item<4-> Aunque correlación menor, se debe a diferencias en escalas
        \item<5-> R² del GARCH indica ajuste modesto pero positivo
    \end{itemize}
\end{frame}

\begin{frame}{Pruebas de Diagnóstico}
    \begin{table}
        \centering
        \begin{tabular}{lccc}
            \toprule
            \textbf{Modelo} & \textbf{Shapiro-Wilk} & \textbf{Durbin-Watson} & \textbf{Breusch-Pagan} \\
            & \textbf{(normalidad)} & \textbf{(autocorr.)} & \textbf{(homoced.)} \\
            \midrule
            Volatilidad Implícita & $< 0.001$ & 0.0118 & $< 0.001$ \\
            Modelo GARCH(1,1) & $< 0.001$ & 0.4405 & $< 0.001$ \\
            \bottomrule
        \end{tabular}
        \caption{Pruebas estadísticas de diagnóstico}
    \end{table}
    
    \vspace{0.5em}
    
    \begin{itemize}
        \item<2-> \textbf{Normalidad}: Ambos modelos rechazan (colas pesadas)
        \item<3-> \textbf{Autocorrelación}: GARCH muestra menor evidencia (0.4405 vs 0.0118)
        \item<4-> \textbf{Homocedasticidad}: Ambos presentan heterocedasticidad residual
    \end{itemize}
\end{frame}