\section{Resultados}

\begin{frame}{Parámetros Estimados GARCH(1,1)}
    \begin{table}
        \centering
        \begin{tabular}{lccc}
            \toprule
            \textbf{Parámetro} & \textbf{Estimación} & \textbf{Error Estándar} & \textbf{Valor-p} \\
            \midrule
            $\mu$ (Media) & 0.0975 & 0.0309 & 0.0016 \\
            $\omega$ & 0.0458 & 0.0252 & 0.0691 \\
            $\alpha_1$ & 0.1076 & 0.0336 & 0.0013 \\
            $\beta_1$ & 0.8383 & 0.0468 & $< 0.001$ \\
            \bottomrule
        \end{tabular}
        \caption{Parámetros del modelo GARCH(1,1) para S\&P 500}
    \end{table}
    
    \vspace{0.5em}
    
    \begin{itemize}
        \item<2-> \highlight{Alta persistencia}: $\alpha_1 + \beta_1 = 0.9459$
        \item<3-> Parámetros significativos al 5\% (excepto $\omega$ marginalmente)
        \item<4-> Condición de estacionaridad se cumple
        \item<5-> Log-likelihood: -813.264 con 635 observaciones
    \end{itemize}
\end{frame}

\begin{frame}{Interpretación Económica}
    \begin{block}{Persistencia de Volatilidad}
        Coeficiente $\alpha_1 + \beta_1 = 0.9459$ indica:
        \begin{itemize}
            \item Choques de volatilidad tienen \emphasis{efectos duraderos}
            \item Vida media $\approx \frac{\ln(0.5)}{\ln(0.9459)} \approx 12$ períodos
            \item Consistente con clustering de volatilidad
        \end{itemize}
    \end{block}
    
    \vspace{0.5em}
    
    \begin{block}{Componentes de Volatilidad}
        \begin{itemize}
            \item<2-> $\omega = 0.0458$: nivel base de volatilidad
            \item<3-> $\alpha_1 = 0.1076$: efecto inmediato (ARCH)
            \item<4-> $\beta_1 = 0.8383$: dependencia del pasado (GARCH)
        \end{itemize}
    \end{block}
\end{frame}

\begin{frame}{Comparación Visual: GARCH vs Volatilidad Histórica}
    \begin{figure}
        \centering
        \includegraphics[width=0.7\textwidth]{../images/volatilidad_seaborn_comparacion.png}
        \caption{Volatilidad histórica vs predicción GARCH(1,1)}
    \end{figure}
        
    \begin{itemize}
        \item<2-> GARCH captura efectivamente los \emphasis{clusters de volatilidad}
        \item<3-> Correspondencia razonable durante períodos de alta volatilidad
    \end{itemize}
\end{frame}

\begin{frame}{Criterios de Información}
    \begin{table}
        \centering
        \begin{tabular}{lccc}
            \toprule
            \textbf{Modelo} & \textbf{AIC} & \textbf{BIC} & \textbf{Log-Likelihood} \\
            \midrule
            Volatilidad Implícita (VIX) & 893.91 & 898.36 & -445.96 \\
            Modelo GARCH(1,1) & 886.33 & 904.11 & -439.16 \\
            \bottomrule
        \end{tabular}
        \caption{Comparación de criterios de información}
    \end{table}
    
    \vspace{0.5em}
    
    \begin{itemize}
        \item<2-> \highlight{Competencia cerrada entre modelos}:
        \begin{itemize}
            \item GARCH ventaja en AIC: 7.58 puntos
            \item VIX ventaja en BIC: 5.76 puntos
        \end{itemize}
        \item<3-> Ambos modelos capturan aspectos complementarios
    \end{itemize}
\end{frame}

\begin{frame}{Fórmulas de Evaluación}
    \begin{itemize}
        \item \textbf{MAE}:
        {\footnotesize $$MAE = \frac{1}{n}\sum_{i=1}^{n}|y_i - \hat{y}_i|$$}
  
        \item \textbf{RMSE}:
        {\footnotesize $$RMSE = \sqrt{\frac{1}{n}\sum_{i=1}^{n}(y_i - \hat{y}_i)^2}$$}
  
        \item \textbf{MAPE}:
        {\footnotesize $$MAPE = \frac{100}{n}\sum_{i=1}^{n}\left|\frac{y_i - \hat{y}_i}{y_i}\right|$$}
  
        \item \textbf{R²}:
        {\footnotesize $$R^2 = 1 - \frac{SS_{res}}{SS_{tot}} = 1 - \frac{\sum_{i=1}^{n}(y_i - \hat{y}_i)^2}{\sum_{i=1}^{n}(y_i - \bar{y})^2}$$}
  
        \item \textbf{Correlación}:
        {\footnotesize $$r = \frac{\sum_{i=1}^{n}(x_i - \bar{x})(y_i - \bar{y})}{\sqrt{\sum_{i=1}^{n}(x_i - \bar{x})^2 \sum_{i=1}^{n}(y_i - \bar{y})^2}}$$}
    \end{itemize}
\end{frame}

\begin{frame}{Métricas de Precisión}
    \begin{table}
        \centering
        \footnotesize
        \begin{tabular}{lccccc}
            \toprule
            \textbf{Modelo} & \textbf{MAE} & \textbf{RMSE} & \textbf{MAPE(\%)} & \textbf{R²} & \textbf{Correlación} \\
            \midrule
            Volatilidad Implícita & 0.3201 & 0.4911 & 38.35 & 0.1320 & 0.5430 \\
            Modelo GARCH(1,1) & 0.2852 & 0.4859 & 38.58 & 0.1505 & 0.4488 \\
            \bottomrule
        \end{tabular}
        \caption{Métricas de precisión en predicción de volatilidad}
    \end{table}
    
    \vspace{0.5em}
    
    \begin{itemize}
        \item<2-> \highlight{Desempeño competitivo} entre ambos enfoques
        \item<3-> GARCH: MAE ligeramente menor (0.2852 vs 0.3201)
        \item<4-> VIX: Correlación superior (0.5430 vs 0.4488)
        \item<5-> MAPE similar: GARCH 38.58\% vs VIX 38.35\%
    \end{itemize}
\end{frame}

\begin{frame}{Pruebas de Diagnóstico}
    \begin{table}
        \centering
        \begin{tabular}{lccc}
            \toprule
            \textbf{Modelo} & \textbf{Shapiro-Wilk} & \textbf{Durbin-Watson} & \textbf{Breusch-Pagan} \\
            & \textbf{(normalidad)} & \textbf{(autocorr.)} & \textbf{(homoced.)} \\
            \midrule
            Volatilidad Implícita & $< 0.001$ & 0.3038 & $< 0.001$ \\
            Modelo GARCH(1,1) & $< 0.001$ & 0.2973 & $< 0.001$ \\
            \bottomrule
        \end{tabular}
        \caption{Pruebas estadísticas de diagnóstico}
    \end{table}
    
    \vspace{0.5em}
    
    \begin{itemize}
        \item<2-> \textbf{Normalidad}: Ambos modelos rechazan (colas pesadas)
        \item<3-> \textbf{Autocorrelación}: GARCH menor evidencia (0.2973 vs 0.3038)
        \item<4-> \textbf{Homocedasticidad}: Ambos presentan heterocedasticidad residual
    \end{itemize}
\end{frame}

\begin{frame}{Comparación VIX vs Volatilidad Histórica}
    \begin{figure}
        \centering
        \includegraphics[width=0.7\textwidth]{../images/volatilidad_vix_comparacion.png}
        \caption{Volatilidad histórica vs volatilidad implícita (VIX)}
    \end{figure}
        
    \begin{itemize}
        \item<2-> \emphasis{Correlación significativa} de 0.5430 entre series
        \item<3-> VIX muestra tendencia a \highlight{anticipar movimientos}
        \item<4-> Complementariedad entre medidas prospectivas y retrospectivas
    \end{itemize}
\end{frame}

\begin{frame}{Validación Temporal: Predicciones Diarias}
    \begin{table}
        \centering
        \tiny
        \begin{tabular}{lcccccc}
            \toprule
            \textbf{Fecha} & \textbf{Día} & \textbf{Vol. Real} & \textbf{GARCH} & \textbf{VIX} & \textbf{Error GARCH} & \textbf{Error VIX} \\
            \midrule
            21 jul & Lun & 0.3346 & 0.7021 & 0.1665 & 0.3675 & 0.1681 \\
            22 jul & Mar & 0.2885 & 0.6812 & 0.1650 & 0.3927 & 0.1235 \\
            23 jul & Mié & 0.3836 & 0.6597 & 0.1537 & 0.2762 & 0.2299 \\
            24 jul & Jue & 0.2912 & 0.6996 & 0.1539 & 0.4084 & 0.1373 \\
            25 jul & Vie & 0.2948 & 0.6749 & 0.1493 & 0.3801 & 0.1455 \\
            \midrule
            \textbf{MAE} & & & \textbf{0.3650} & \textbf{0.1609} & & \\
            \bottomrule
        \end{tabular}
        \caption{Validación prospectiva (21-25 julio 2025)}
    \end{table}
    
    \vspace{0.2em}
    
    \begin{itemize}
        \item<2-> VIX demostró \highlight{superioridad del 55.9\%} en precisión
        \item<3-> Mejor desempeño en 4 de 5 días analizados
        \item<4-> Capacidad forward-looking captura expectativas de mercado
    \end{itemize}
\end{frame}