\section{Introducción}

\begin{frame}{Introducción}
    \begin{itemize}
        \item<1-> Los mercados financieros presentan \emphasis{volatilidad cambiante en el tiempo}
        \item<2-> La volatilidad es un factor clave en:
        \begin{itemize}
            \item Gestión de riesgos
            \item Valoración de derivados
            \item Decisiones de inversión
        \end{itemize}
        \item<3-> Los modelos GARCH capturan la \highlight{heterocedasticidad condicional}
        \item<4-> Aplicación práctica: análisis del S\&P 500 vs volatilidad implícita (VIX)
    \end{itemize}
\end{frame}

\begin{frame}{Motivación}
    \begin{columns}
        \begin{column}{0.6\textwidth}
            \begin{itemize}
                \item<1-> Las series de tiempo financieras no cumplen condiciones de estacionaridad
                \item<2-> La volatilidad (varianza) no se mantiene constante en el tiempo
                \item<3-> ¿Es posible modelar la heterocedasticidad de las series temporales?
                \item<4-> Los modelos ARCH/GARCH responden esta pregunta
            \end{itemize}
        \end{column}
        \begin{column}{0.4\textwidth}
            \begin{figure}
                \centering
                \includegraphics[width=\textwidth]{../images/series_temporales.png}
                \caption{\footnotesize S\&P 500 y VIX}
            \end{figure}
        \end{column}
    \end{columns}
\end{frame}

\begin{frame}{Objetivos}
    \begin{block}{Objetivo General}
        Implementar y evaluar modelos GARCH(1,1) para el modelado de volatilidad del S\&P 500, comparando su desempeño con la volatilidad implícita (VIX)
    \end{block}
    
    \vspace{0.5em}
    
    \begin{block}{Objetivos Específicos}
        \begin{enumerate}
            \item Desarrollar fundamentos teóricos de modelos ARCH/GARCH
            \item Implementar modelo GARCH(1,1) para retornos del S\&P 500
            \item Comparar predicciones de volatilidad GARCH vs VIX
            \item Evaluar bondad de ajuste mediante criterios estadísticos
        \end{enumerate}
    \end{block}
\end{frame}