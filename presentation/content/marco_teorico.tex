\section{Marco Teórico}

\begin{frame}{Heterocedasticidad Condicional}
    \begin{itemize}
        \item<1-> En series financieras: $\text{Var}(r_t | \Fcal_{t-1}) = \sigma_t^2$
        \item<2-> La varianza condicional \emphasis{cambia en el tiempo}
        \item<3-> Características estilizadas:
        \begin{itemize}
            \item Clusters de volatilidad
            \item Colas pesadas
            \item Asimetría en respuesta a shocks
        \end{itemize}
    \end{itemize}
    
    \vspace{1em}
    
    \onslide<4->{
    \begin{alertblock}{Problema}
        Los modelos clásicos asumen varianza constante: $\sigma^2$
    \end{alertblock}
    }
\end{frame}

\begin{frame}{Modelo GARCH(p,q)}
    \begin{block}{Especificación}
        \begin{align}
            r_t &= \mu_t + \epsilon_t \\
            \epsilon_t &= \sigma_t z_t, \quad z_t \sim \text{i.i.d.}(0,1) \\
            \sigma_t^2 &= \omega + \sum_{i=1}^q \alpha_i \epsilon_{t-i}^2 + \sum_{j=1}^p \beta_j \sigma_{t-j}^2
        \end{align}
    \end{block}
    
    \vspace{0.5em}
    
    \begin{itemize}
        \item<2-> $\omega > 0$, $\alpha_i \geq 0$, $\beta_j \geq 0$
        \item<3-> Condición de estacionaridad: $\sum_{i=1}^q \alpha_i + \sum_{j=1}^p \beta_j < 1$
        \item<4-> GARCH(1,1): $\sigma_t^2 = \omega + \alpha \epsilon_{t-1}^2 + \beta \sigma_{t-1}^2$
    \end{itemize}
\end{frame}

\begin{frame}{Extensiones del Modelo GARCH}
    \begin{columns}
        \begin{column}{0.5\textwidth}
            \begin{block}{GJR-GARCH}
                \begin{align}
                    \sigma_t^2 = \omega &+ \alpha \epsilon_{t-1}^2 \\
                    &+ \gamma \epsilon_{t-1}^2 I_{t-1} \\
                    &+ \beta \sigma_{t-1}^2
                \end{align}
                donde $I_{t-1} = 1$ si $\epsilon_{t-1} < 0$
            \end{block}
        \end{column}
        
        \begin{column}{0.5\textwidth}
            \begin{block}{EGARCH}
                \begin{align}
                    \ln(\sigma_t^2) = \omega &+ \alpha \left| \frac{\epsilon_{t-1}}{\sigma_{t-1}} \right| \\
                    &+ \gamma \frac{\epsilon_{t-1}}{\sigma_{t-1}} \\
                    &+ \beta \ln(\sigma_{t-1}^2)
                \end{align}
            \end{block}
        \end{column}
    \end{columns}
    
    \vspace{1em}
    
    \begin{itemize}
        \item<2-> Capturan el \emphasis{efecto apalancamiento}
        \item<3-> Shocks negativos $\rightarrow$ mayor volatilidad
    \end{itemize}
\end{frame}