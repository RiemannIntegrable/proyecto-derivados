\section{Marco Teórico}

\begin{frame}{Series de Tiempo Financieras}
    \begin{itemize}
        \item<1-> Una serie de tiempo: observaciones indexadas en el tiempo
        \item<2-> Modelos ARMA capturan comportamiento bajo \emphasis{estacionaridad}
        \item<3-> En finanzas: los precios \highlight{NO son estacionarios}
        \item<4-> Sea $Z_t := \ln\left(\frac{P_t}{P_{t-1}}\right)$ el retorno logarítmico
    \end{itemize}
    
    \vspace{1em}
    
    \onslide<5->{
    \begin{alertblock}{Problema}
        La volatilidad (varianza) no se mantiene constante en el tiempo
    \end{alertblock}
    }
\end{frame}

\begin{frame}{Modelo ARCH}
    \begin{block}{Especificación ARCH(p)}
        Sea $h_t= \mathbb{V}\left(Z_t|Z_s,s<t\right)$, entonces:
        \begin{align}
            Z_t &= \sqrt{h_t} e_{t} \\
            h_t &= \alpha_0 + \sum_{i=1}^{p} \alpha_i Z_{t-i}^2
        \end{align}
        donde $\{e_t\}$ es Normal$(0,1)$ i.i.d.
    \end{block}
    
    \vspace{0.5em}
    
    \begin{itemize}
        \item<2-> $\alpha_0>0$ y $\alpha_j\geq 0$ para $j\in \{1,\cdots,p\}$
        \item<3-> Para ARCH(1): si $\alpha_1<1$ existe solución estacionaria
        \item<4-> $\mathbb{V}(Z_t) = \frac{\alpha_0}{1-\alpha_1}$
    \end{itemize}
\end{frame}

\begin{frame}{Modelo GARCH(p,q)}
    \begin{block}{Generalización del ARCH}
        \begin{align}
            Z_t &= \sqrt{h_t} e_{t} \\
            h_t &= \alpha_0 + \sum_{i=1}^{p} \alpha_i Z_{t-i}^2 + \sum_{j=1}^{q} \beta_j h_{t-j}
        \end{align}
    \end{block}
    
    \vspace{0.5em}
    
    \begin{itemize}
        \item<2-> $\alpha_0 > 0$, $\alpha_i \geq 0$, $\beta_j \geq 0$
        \item<3-> Estacionaridad: $\sum_{i=1}^{\max(p,q)} (\alpha_i + \beta_i) < 1$
        \item<4-> GARCH(1,1): $h_t = \omega + \alpha Z_{t-1}^2 + \beta h_{t-1}$
    \end{itemize}
\end{frame}

\begin{frame}{Solución Analítica GARCH(1,1)}
    \begin{block}{Substitución Recursiva}
        Asumiendo $\alpha + \beta < 1$:
        \begin{align}
            h_t = \frac{\omega}{1-\beta} + \alpha \sum_{j=0}^{\infty} \beta^j Z_{t-1-j}^2
        \end{align}
    \end{block}
    
    \vspace{0.5em}
    
    \begin{itemize}
        \item<2-> Volatilidad = media ponderada exponencialmente decreciente
        \item<3-> Velocidad de decaimiento determinada por $\beta$
        \item<4-> Varianza incondicional: $\sigma^2 = \frac{\omega}{1 - \alpha - \beta}$
    \end{itemize}
    
    \vspace{0.5em}
    
    \onslide<5->{
    \begin{block}{Propiedades}
        \begin{itemize}
            \item Persistencia: $\alpha + \beta$ cerca de 1 $\Rightarrow$ efectos duraderos
            \item Clustering de volatilidad
            \item Vida media de choques: $\frac{\ln(0.5)}{\ln(\alpha + \beta)}$
        \end{itemize}
    \end{block}
    }
\end{frame}

\begin{frame}{Volatilidad Implícita - VIX}
    \begin{itemize}
        \item<1-> \textbf{Volatilidad histórica}: basada en movimientos pasados
        \item<2-> \textbf{Volatilidad implícita}: expectativas del mercado (forward-looking)
        \item<3-> \textbf{VIX}: índice de volatilidad implícita del S\&P 500
    \end{itemize}
    
    \vspace{0.5em}
    
    \onslide<4->{
    \begin{block}{Fórmula del VIX}
        \begin{align}
            VIX = 100 \times \sqrt{\frac{2}{T} \sum_i \frac{\Delta K_i}{K_i^2} e^{rT} Q(K_i) - \frac{1}{T}\left[\frac{F}{K_0} - 1\right]^2}
        \end{align}
    \end{block}
    }
    
    \vspace{0.5em}
    
    \onslide<5->{
    \begin{itemize}
        \item \highlight{"Barómetro del miedo"} del mercado
        \item Correlación negativa con retornos del S\&P 500
        \item Mean reversion hacia $\approx 20\%$
    \end{itemize}
    }
\end{frame}