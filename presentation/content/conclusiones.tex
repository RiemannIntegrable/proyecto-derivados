\section{Conclusiones}

\begin{frame}{Conclusiones Principales}
    \begin{enumerate}
        \item<1-> \textbf{Heterocedasticidad Condicional}
        \begin{itemize}
            \item Evidencia estadística fuerte en series del COLCAP
            \item Justifica el uso de modelos GARCH
        \end{itemize}
        
        \item<2-> \textbf{Efectos Asimétricos}
        \begin{itemize}
            \item GJR-GARCH supera al GARCH estándar
            \item Shocks negativos generan mayor volatilidad
        \end{itemize}
        
        \item<3-> \textbf{Alta Persistencia}
        \begin{itemize}
            \item Coeficiente de persistencia $\approx 0.96$
            \item Shocks de volatilidad tienen efectos duraderos
        \end{itemize}
        
        \item<4-> \textbf{Aplicaciones Prácticas}
        \begin{itemize}
            \item Gestión de riesgos
            \item Valoración de opciones
            \item Estrategias de cobertura
        \end{itemize}
    \end{enumerate}
\end{frame}

\begin{frame}{Limitaciones y Trabajo Futuro}
    \begin{columns}
        \begin{column}{0.5\textwidth}
            \begin{block}{Limitaciones}
                \begin{itemize}
                    \item Una sola serie de tiempo
                    \item Período de análisis limitado
                    \item Distribución normal asumida
                    \item Efectos de cambio estructural
                \end{itemize}
            \end{block}
        \end{column}
        
        \begin{column}{0.5\textwidth}
            \begin{block}{Trabajo Futuro}
                \begin{itemize}
                    \item Modelos multivariados (DCC-GARCH)
                    \item Distribuciones de colas pesadas
                    \item Modelos de cambio de régimen
                    \item Validación out-of-sample
                \end{itemize}
            \end{block}
        \end{column}
    \end{columns}
    
    \vspace{1em}
    
    \begin{alertblock}{Impacto}
        Los resultados proporcionan \emphasis{herramientas cuantitativas} para la toma de decisiones en mercados financieros colombianos
    \end{alertblock}
\end{frame}

\begin{frame}{Referencias Principales}
    \begin{thebibliography}{9}
        \bibitem{engle1982}
        Engle, R.F. (1982). Autoregressive conditional heteroscedasticity with estimates of the variance of United Kingdom inflation. \textit{Econometrica}, 50(4), 987-1007.
        
        \bibitem{bollerslev1986}
        Bollerslev, T. (1986). Generalized autoregressive conditional heteroskedasticity. \textit{Journal of Econometrics}, 31(3), 307-327.
        
        \bibitem{glosten1993}
        Glosten, L.R., Jagannathan, R., \& Runkle, D.E. (1993). On the relation between the expected value and the volatility of the nominal excess return on stocks. \textit{Journal of Finance}, 48(5), 1779-1801.
        
        \bibitem{nelson1991}
        Nelson, D.B. (1991). Conditional heteroskedasticity in asset returns: A new approach. \textit{Econometrica}, 59(2), 347-370.
    \end{thebibliography}
\end{frame}