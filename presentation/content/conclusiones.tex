\section{Conclusiones}

\begin{frame}{Conclusiones Principales}
    \begin{enumerate}
        \item<1-> \textbf{Modelo GARCH(1,1) Exitoso}
        \begin{itemize}
            \item Alta persistencia de volatilidad ($\alpha_1 + \beta_1 = 0.9459$)
            \item Parámetros estadísticamente significativos
            \item Captura efectivamente clusters de volatilidad
        \end{itemize}
        
        \item<2-> \textbf{Superioridad sobre VIX}
        \begin{itemize}
            \item GARCH supera significativamente en criterios AIC/BIC
            \item Errores de predicción considerablemente menores
            \item MAPE: 49.57\% vs 2424.08\% del VIX
        \end{itemize}
        
        \item<3-> \textbf{Implicaciones Económicas}
        \begin{itemize}
            \item Choques de volatilidad tienen efectos duraderos ($\approx$ 12 períodos)
            \item Modelos econométricos complementan información de mercado
            \item Vida media de choques consistente con literatura financiera
        \end{itemize}
        
        \item<4-> \textbf{Aplicaciones Prácticas}
        \begin{itemize}
            \item Gestión de riesgo (VaR, Expected Shortfall)
            \item Mejora en pricing de derivados (Black-Scholes)
            \item Optimización de portafolios y trading algorítmico
        \end{itemize}
    \end{enumerate}
\end{frame}

\begin{frame}{Limitaciones y Extensiones}
    \begin{columns}
        \begin{column}{0.5\textwidth}
            \begin{block}{Limitaciones Identificadas}
                \begin{itemize}
                    \item Violación del supuesto de normalidad
                    \item Heterocedasticidad residual persistente
                    \item No captura efectos de apalancamiento
                    \item Limitaciones en eventos extremos
                \end{itemize}
            \end{block}
        \end{column}
        
        \begin{column}{0.5\textwidth}
            \begin{block}{Extensiones Futuras}
                \begin{itemize}
                    \item Modelos GJR-GARCH o EGARCH
                    \item Distribuciones de colas pesadas (t-Student)
                    \item Modelos multivariados (DCC-GARCH)
                    \item Combinación GARCH + VIX
                \end{itemize}
            \end{block}
        \end{column}
    \end{columns}
    
    \vspace{1em}
    
    \begin{alertblock}{Contribución}
        Demostración empírica de que modelos GARCH pueden \emphasis{superar} la volatilidad implícita del mercado en ciertas condiciones
    \end{alertblock}
\end{frame}

\begin{frame}{Referencias}
    \begin{thebibliography}{9}
        \bibitem{bollerslev1986}
        Bollerslev, T. (1986). Generalized autoregressive conditional heteroskedasticity. \textit{Journal of Econometrics}, 31(3), 307-327.
        
        \bibitem{engle1982}
        Engle, R.F. (1982). Autoregressive conditional heteroscedasticity with estimates of the variance of United Kingdom inflation. \textit{Econometrica}, 50(4), 987-1007.
        
        \bibitem{cboe2019}
        Chicago Board Options Exchange (2019). \textit{VIX White Paper}. CBOE.
        
        \bibitem{tsay2010}
        Tsay, R.S. (2010). \textit{Analysis of Financial Time Series}. 3rd Edition, Wiley.
    \end{thebibliography}
\end{frame}